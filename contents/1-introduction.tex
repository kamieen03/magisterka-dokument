\section{Introduction}
\subsection{Background}
The last couple \WK{of} years saw a rapidly surging interest in neural networks
invariant to certain groups of transformations. This property
is very much sought after because under mild assumptions it
guarantees the output of the network won't change when transformed according to
chosen
transformations. Related desirable quality of neural networks is equivariance
which on the other hand causes output to change in the same way the input
changes. Growing body of work presents wide variety of
techniques applied to ensure these properties.
\\ However so far almost all of published papers focused exclusively on
geometric symmetries of images, e.g. rotations, scaling or reflection. Transforms
related to lightning of the image, like contrast or brightness have been largely
ignored.
\subsection{Objectives}
In this work we address the following problems:
\begin{enumerate}
    \item Is constructing neural networks either equivariant or invariant to
        lightning symmetries feasible? If so then how do they compare to networks
        lacking these properties?
    \item What's the degree of equivariance of such models? Is the equivariance
        exact or does inevitable discretization impose significant error?
        Does it differ significantly from equivariance to geometric
        transformations? \WK{last sentence unclear}
\end{enumerate}
In order to answer these questions, we carry out the following tasks:
\begin{enumerate}
    \item Possibly extending notions of contrast, brightness, gamma correction 
        and color balance from
        images to tensors of arbitrary dimensionality.
    \item Constructing neural network layers invariant to various lightning
        symmetries.
    \item Adapting existing architectures to ensure equivariance to some or
        possibly all of mentioned lightning transformations.
    \item Comparing constructed models with control group on image
        classification tasks.
    \item Estimating numerically degree of invariance and equivariance of constructed models.
\end{enumerate}
